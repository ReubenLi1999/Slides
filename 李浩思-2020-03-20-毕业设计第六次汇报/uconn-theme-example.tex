\documentclass{beamer}

\usepackage[utf8]{inputenc}
\usepackage[T1]{fontenc}

\usetheme[campus=health,watermark=side,logo=stacked]{uconn}

\title[Beamer themes for UConn]{Beamer themes for the\\ University of Connecticut}
\subtitle{part of a suite of themes}
\author[Cory Brunson]{Jason Cory Brunson, PhD}
\institute[UConn Health]{Center for Quantitative Medicine\\ University of Connecticut School of Medicine}
\date{\today}

\newtheorem{remark}{Remark}

\setlength{\parskip}{.5em}

\begin{document}


\begin{frame}
\titlepage
\end{frame}


\begin{frame}{Table of contents}
\tableofcontents
\end{frame}


\begin{frame}{UConn theme}

The {\ttfamily uconn} Beamer theme, stored in the file {\ttfamily beamerthemeuconn.sty}, simply invokes four specific {\ttfamily uconn} Beamer themes, stored in the following files:
\begin{itemize}
\ttfamily
\item beamercolorthemeuconn.sty
\item beamerfontthemeuconn.sty
\item beamerinnerthemeuconn.sty
\item beamerouterthemeuconn.sty
\end{itemize}

This example slideshow is just a concatenation of two example slideshows for the color and font themes ({\ttfamily uconn-color-font-example.tex}) and for the inner and outer themes ({\ttfamily uconn-inner-outer-example.tex}).

\end{frame}


\section{Layout}


\begin{frame}[fragile]{Icons}

The background image is a faded navy blue oak leaf, and the footer includes a UConn Health woodmark, both stored as PNGs.
They are activated by the \verb|watermark| and \verb|logo| options, with some additional options that control the logo:

\begin{verbatim}
\usetheme[
  campus=health,
  watermark=side,
  logo=stacked
]{uconn}
\end{verbatim}

See {\ttfamily uconn-inner-outer-theme-example.tex} for details on the logo options.

Like the others used in this theme, the image comes from the \href{https://brand.uconn.edu/downloads/logos/}{bulk logos download at UConn Brand Standard}.

\end{frame}


\begin{frame}[fragile]{Symbols}

The classic (black) oak leaf is encoded as a symbol \verb|\oakleaf|, and can be used in mathematical environments, e.g.
\[\oakleaf^{\mathbb{C}}=\oakleaf\otimes_{\mathbb{R}}\mathbb{C}\text,\]
or in text (\oakleaf).

An inverted-color oak leaf is encoded as \verb|\oakleafbox| (see the slide on claims and proofs).

\end{frame}


\section{Environments}


\subsection{Text environments}


\begin{frame}{Blocks}

\begin{block}{Block}
This is a text block, formatted as in the {\ttfamily default} template.
\end{block}

\begin{alertblock}{Alert block}
This is an alert text block, also formatted by default.
\end{alertblock}

More may be done in future to customize the block environments. Suggestions are welcome!

\end{frame}


\subsection{Mathematical environments}


\begin{frame}[fragile]{Definitions and examples}

\begin{definition}
A {\bfseries definition} block is by default formatted like a text block.
\end{definition}

Text formatting like \verb|\bfseries| (boldface) can be called within blocks.

\begin{example}
An \emph{example} is also a block, formatted a bit differently.
\end{example}

\end{frame}


\begin{frame}[fragile]{Claim and proof environments}

\begin{theorem}
Claim blocks, including \verb|lemma|, \verb|theorem|, and \verb|corollary|, use slant-shaped font (\verb|\slshape|) to emphasize their contents.
\end{theorem}

\begin{proof}
This proof block shows that the customary open square tombstone has been changed to a UConn oak leaf (white inside a black square).
\end{proof}

\begin{corollary}
The box oak leaf can also be used inline: \oakleafbox
\end{corollary}

\end{frame}


\section{Color and font defaults}


\begin{frame}[fragile]{Default font family}

The default Helvetica font family is provided by the {\ttfamily helvet} package.
It is equivalent to Arial, which is used in the PowerPoint templates (see the \hyperlink{slide:acknowledgments}{Acknowledgments slide}).

Block titles are rendered in \href{http://mirrors.ibiblio.org/CTAN/fonts/montserrat/doc/montserrat-doc.pdf}{{\montserratsb Montserrat semibold}}, a free alternative to \href{https://brand.uconn.edu/standards/fonts/}{the Gothic bold} used in campus-, college-, and school-specific woodmarks provided by the {\ttfamily montserrat} package.

Two shortcuts render text in Montserrat semibold:
\begin{itemize}
\item \verb|\montserratsb| (with no argument)
\item \verb|\campus{}| (with a text argument)
\end{itemize}

\end{frame}


\begin{frame}[fragile]{Palette colors}

The palette colors used by outer themes interpolate between navy blue and white:

\vfill

\begin{columns}
\begin{column}{.5\textwidth}

\centering
\verb|palette|
\vspace{1ex}

\begin{beamercolorbox}[sep=4pt,center]{palette primary}
\usebeamerfont{palette primary}primary
\end{beamercolorbox}

\begin{beamercolorbox}[sep=4pt,center]{palette secondary}
\usebeamerfont{palette secondary}secondary
\end{beamercolorbox}

\begin{beamercolorbox}[sep=4pt,center]{palette tertiary}
\usebeamerfont{palette tertiary}tertiary
\end{beamercolorbox}

\begin{beamercolorbox}[sep=4pt,center]{palette quaternary}
\usebeamerfont{palette quaternary}quaternary
\end{beamercolorbox}

\end{column}
\begin{column}{.5\textwidth}

\centering
\verb|palette sidebar|
\vspace{1ex}

\begin{beamercolorbox}[sep=4pt,center]{palette primary}
\usebeamerfont{palette primary}primary
\end{beamercolorbox}

\begin{beamercolorbox}[sep=4pt,center]{palette secondary}
\usebeamerfont{palette secondary}secondary
\end{beamercolorbox}

\begin{beamercolorbox}[sep=4pt,center]{palette tertiary}
\usebeamerfont{palette tertiary}tertiary
\end{beamercolorbox}

\begin{beamercolorbox}[sep=4pt,center]{palette quaternary}
\usebeamerfont{palette quaternary}quaternary
\end{beamercolorbox}

\end{column}
\end{columns}

\vspace{2ex}
For example, \verb|palette secondary| is defined using
\begin{verbatim}
bg=uconn navy blue!80!uconn white,fg=uconn white
\end{verbatim}

\end{frame}


\begin{frame}{Frame element colors}

The symbol colors for itemized and enumerated lists progress
\begin{itemize}
\item from purple,
\begin{itemize}
\item through orange-gold,
\begin{itemize}
\item to green.
\end{itemize}
\end{itemize}
\end{itemize}
\alert{Alerted text is rendered in orange-gold.}

\vfill\begin{remark}
Block and caption titles are boldfaced and rendered in dark red.
\end{remark}

\vfill
Hyperlinks and buttons are rendered in teal blue.
See \hyperlink{sec:customization}{Customization} and \hyperlink{slide:acknowledgments}{Acknowledgments} for examples.

\end{frame}


\section{Customization}
\label{sec:customization}


\begin{frame}[fragile]{Manually assigning colors}

Type
\begin{verbatim}
{\color{uconn purple} purple}
\end{verbatim}
(with brackets) to render {\color{uconn purple} purple}, or set
\begin{verbatim}
\color{uconn health bright teal}
\end{verbatim}
\color{uconn health bright teal}
(without brackets) to color all text in a slide the bright teal in the UConn Health palette.

\color{black}\vfill
Color names are based on the UConn Brand color palette.

\center\href{http://brand.uconn.edu/standards/color-palette/}{\beamerbutton{Visit the Brand Standards website}}

\end{frame}


\begin{frame}[fragile]{Changing the theme}

You can change the default colors by using setbeamercolor in the preamble, as i did:
\begin{verbatim}
\setbeamercolor*{titlelike}{
  bg=uconn navy blue,
  fg=uconn white
}
\end{verbatim}
\ldots or by editing the color theme file {\ttfamily beamercolorthemeuconn.sty}.
Please \hyperlink{mailto:brand@uconn.edu}{check with the Brands office} about using the derivative theme, e.g.\ one with the colors themselves altered.

\end{frame}


\section{Thanks}


\begin{frame}{Acknowledgments}
\label{slide:acknowledgments}

See the color/font themes example and the inner/outer themes example for specific acknowledgments.

Suggestions are always welcome! Email Cory: \href{mailto:brunson@uchc.edu}{\ttfamily brunson@uchc.edu}

\end{frame}


\end{document}
